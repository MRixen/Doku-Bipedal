\chapter{Einleitung}
%
% Allgemein
%
Bereits vor dem 21. Jahrhundert besteht der Wunsch eine Maschine zu erschaffen, die sich durch ihre Agilit�t und Dynamik im menschlichen Lebensraum problemlos fortbewegen kann, um k�rperlich schwere und stumpfe Arbeiten dem Menschen abzunehmen. Als Vorbild dient der Mensch selbst, da sich dessen k�rperliche Struktur im Laufe der Evolution bew�hrt hat und, von der physischen Seite betrachtet, ausgereift ist.

%
% Ausgangspunkt
%
Es exisiteren viele Arten von Humanoiden Robotern, durch die das Verst�ndnis der Interaktion im menschlichen Lebensraum, sowie der menschlichen Fortbewegung intensiv erforscht wird.\\
W�hrend viele Konstrukte wieder verworfen werden, bleiben andere bestehen, um diese kontinuierlich zu optimieren. Die Analyse und Bewertung der Systeme erfolgt u.A. anhand von Roboter-Events, bei denen verschiedene Tasks autonom ausgef�hrt werden m�ssen. Dabei zeigt sich, dass die Menschen �hnliche Fortbewegung zwar noch nicht ausgereift, jedoch in einem gewissen Bereich gut funktionst�chtig ist. Die technische Umsetzung erfolgt aus Regelalgorithmen, die in ihrem Umfang eingeschr�nkt sind.\\
Aus diesem Grunde wird in dieser Ausarbeitung ein datenbankbasierter Ansatz verfolgt. Das beinhaltet die Aufnahme der Bewegung beteiligter Gliedma�en bei der menschlichen Fortbewegung. Daraus erfolgt das Extrahieren einer Datenbank, sowie das Implementieren eines Algorithmus, der mit diesen Daten die menschliche Fortbewegung nachbilden kann.

%
% Kapitelinhalt
%
Der erste Teil besteht in dem Aufbau eines Messystems, mit dem die Bewegungen aufgenommen und entsprechende Schrittfolgen extrahiert werden k�nnen (s. Kapitel \ref{kap:Messsystem}). Im zweiten Teil erfolgt der Aufbau eines mechanischen Systems und die Entwicklung eines Movement-Control-Algorithmus (s. Kapitel  \ref{kap:Robotersystem}):