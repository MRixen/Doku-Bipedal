\noindent
\begin{minipage}{\textwidth}
\chapter{Zusammenfassung}
\label{sec:Zusammenfassung}
Zun�chst soll ein �berblick gegeben werden, der die wichtigsten Aspekte des gesamten Systems beschreibt (s. Tabell \ref{tab:AspekteGesamtsystem}).

\begin{center}
\begin{tabular}{|p{5cm}|p{3cm}|p{3cm}|}
	\hline 
	\textbf{Beschreibung} & \textbf{Kontext} & \textbf{Begr�ndung} \\ 
	\hline 
	Es k�nnen max. 8 Motoren per Zyklus angesprochen werden & Ansteuerung Robotersystem & Probleme Zusammensetzen mehrerer Bytes \\ 
	\hline 
	Die Datenbankgr��e darf f�r den Zeitstempel max. 65535 (2 byte) aufweisen & Ansteuerung Robotersystem & Aktuelle Implementierung Protokoll \\ 
	\hline 
	Das System l�sst nur das aktive Steuern zu. Autonomie ist nicht gegeben. &  Ansteuerung Robotersystem & Funktionsumfang aktueller Version \\ 
	\hline 
	&  &  \\ 
	\hline 
\end{tabular} 
\end{center}

Da eine k�nstliche Intelligenz zu unvorhersehbaren, gef�hrlichen Situationen f�hren kann, muss ein, dem Menschen �hnlich dynamisches System, in seiner Freiheit beschr�nkt werden. Der Begriff Freiheit definiert in diesem Kontext das eigenst�ndige Lernen ohne menschliche Kontrolle. \\
Ein datenbankbasiertes System besitzt hingegen nur die F�higkeiten, welche in der Datenbank hinterlegt sind. Dies bringt den Vorteil der Berechenbarkeit mit sich.
\vspace{2.5cm}
\chapter{Ausblick}
\label{sec:Ausblick}
 Stark unausgereift hingegen ist das zusammenh�ngende Verarbeiten visueller Eindr�cke, auditiver Wahrnehmungen, sowie vestibul�ren Werten und sensomotorischen F�higkeiten.
\end{minipage}
