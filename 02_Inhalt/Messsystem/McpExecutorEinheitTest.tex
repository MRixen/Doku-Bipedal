\subsection{Testversion}
\label{kap:McpExecutorTestversion}
Um die Funktionen der elektronischen Bauteile zu verstehen und beherrschen zu k�nnen erfolgt der Aufbau einer Einheit auf einem Entwicklungsboard (s. Abbildung \ref{fig:McpExecutorTest}). Dieser Aufbau beinhaltet alles, was f�r eine Einheit vorgesehen ist, d.h. einen Atmega328 (mit externem Taktgeber), einen Adxl345 und einen Mcp2515. Mit entsprechender Verdrahtung (s. Abbildung \ref{fig:SchaltbildEinheit}) und der Entwicklung eines Programms, welches auf der MCU aktiv ist, lassen sich die Sensorwerte �ber einen CAN-Bus auslesen. Die Sensor-ID muss programmatisch auf der MCU definiert werden. 

Mit zwei solcher Testaufbauten wurde das Aufnehmen von Sensorwerten validiert.

\begin{figure}[H]
	\centering
	\includegraphics[width=0.9\linewidth]{03_Grafiken/Messsystem/McpExecutorTest}
	\caption[Testversion einer Mcp-Executor-Einheit]{Testversion einer Mcp-Executor-Einheit}
	\label{fig:McpExecutorTest}
\end{figure}

\begin{figure}[H]
\centering
\includegraphics[width=0.7\linewidth]{03_Grafiken/pseudoImage}
\caption[Schaltbild Mcp-Executor Einheit]{Schaltbild Mcp-Executor Einheit}
\label{fig:SchaltbildEinheit}
\end{figure}

