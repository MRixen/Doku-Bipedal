\subsection{Verwendete Komponenten}
\label{kap:HauptrechnerVerwendeteKomponenten}
Die Bestandteile des Hauptrechners sind ein Raspberry-Pi 2 mit Wlan-Adapter und dem Betriebsystem Windows 10 IoT, sowie einer Energiequelle (Akku von Anker) mit 20100 mAh.

\begin{itemize}
	\item \textbf{Raspberry-Pi 2}
	
	Der Raspberry Pi 2 (s. Abbildung \ref{fig:Raspberry}) ist in seiner zweiten Version ein stabiles System, welches in Verbindung mit Windows IoT in der HOchsprache C\# programmiert werden kann. Die Programme entsprechen den Anwendungen wie man sie von Smartphones kennt.
	\newline
	Die zweite Version beinhaltet (im Vergleich zur Version 3) keinen Wlan-Adapter, weshalb dieser zus�tzlich erworben werden muss.
	
	%\begin{figure}[H]
	%	\centering
	%	\includegraphics[width=0.7\linewidth]{03_Grafiken/Messsystem/Raspberry}
	%	\caption[Raspberry Pi 2]{Raspberry Pi 2}
	%	\label{fig:Raspberry}
	%\end{figure}

	\item \textbf{Wlan-Adapter}
	
	Bei der Wahl eines Wlan-Adapters muss die Kompatibilit�t zu Windows-IoT ber�cksichtigt werden, da nicht f�r jedes Ger�t entsprechende Treiber zur Verf�gung stehen. Aus diesem Grunde liegt die Wahl bei dem preisg�nstigen Wlan-Adapter XXXX (s. Abbildung \ref{fig:Adxl345}).
	
	%\begin{figure}[H]
	%	\centering
	%	\includegraphics[width=0.7\linewidth]{03_Grafiken/Messsystem/WlanAdapter}
	%	\caption[Wlan Adapter]{Wlan Adapter}
	%	\label{fig:WlanAdapter}
	%\end{figure}
	
	\item \textbf{Akku von Anker}
	
	F�r die Energieversorgung (s. Abbildung \ref{fig:Anker}) wird ein Akku ben�tigt, welcher eine hohe Kapazit�t aufweist und gen�gend Strom liefert, damit das System ordnungsgem�� funktioniert. Bspw. ben�tigt der Raspberry Pi 2 2A, um einen fehlerfreien Betrieb zu gew�hrleisten (mit zus�tzlich angeschlossener Hardware, wie z.B. der Wlan-Adapter).  
	\newline
	Der Akku PowerCore 20100 von Anker ist f�r diese Anforderungen wie geschaffen, da dieser 20100 mAh aufweist und 2 USB-Ports mit jeweils 2A zur Verf�gung stellt.
		
	%\begin{figure}[H]
	%	\centering
	%	\includegraphics[width=0.7\linewidth]{03_Grafiken/Messsystem/Anker}
	%	\caption[Akku Anker]{Akku Anker}
	%	\label{fig:Anker}
	%\end{figure}

	
\end{itemize}