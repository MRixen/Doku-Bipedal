\subsection{Programmaufbau}
\label{kap:McpExecutorProgrammaufbau}
\begin{enumerate}
	%
	% Setup-Routine
	%
	\item \textbf{Setup-Routine}\\
	% LISTING F�R SETUP() 
	Das Programm, welches auf der \gls{MCU} einer Einheit aktiv ist besteht aus einer Setup- und einer Loop-Routine. Innerhalb des Setups efolgt die Definition verschiedener Variablen und die Konfiguration der SPI-Schnisttstelle, sowie der Komponenten Adxl345 und Mcp2515 (s. Listing \ref{lst:mcpExecutorSetup}).
	In den Zeilen 4-6 und 8-10 werden die Ausg�nge zum Ausw�hlen der Ger�te, entsprechend der SPI-Schnittstelle, gesetzt. Dabei ist zu beachten, dass bei der Atmega-\gls{MCU} der Ausgang an Pin 10 immer immer gesetzt werden muss, auch wenn ein anderer Pin als Device-Selector verwendet wird. Der SIgnalzustand der Pins muss HIGH entsprechen, da bei einem Low-Pegel das entsprechende Ger�t ausgew�hlt wird.\\
	Zeile 12-13 initialisiert die SPI-Schnittstelle mit 5 Mhz und dem SPI-Mode 3 (CPOL = 1, CPHA = 1). Diese einzig verf�gbare Einstellung richtet sich nach dem Beschleunigungssensor Adxl345. Mit dem Aufruf der Routine SPI.begin() beginnt die Verwendung der Schnittstelle, sodass in den Zeilen 15-16 die Initialisierungsdaten an die Ger�te gesendet werden k�nnen. Diese beinhalten f�r den Sensor Adxl345 u.A. das Datenformat und f�r den Mcp2515 das CAN-Bus Setup. F�r das CAN-Bus Setup ist zu beachten, dass die Nachrichten jedes Teilnehmers einen spzifischen Identifier aufweisen muss.\\
	\label{lst:mcpExecutorSetup}
\begin{lstlisting}[language=Java, caption=Setup Routine]
void setup()
{
...
pinMode(CS_PIN_ADXL, OUTPUT);
pinMode(CS_PIN_MCP2515, OUTPUT);
pinMode(10, OUTPUT);

digitalWrite(CS_PIN_ADXL, HIGH);
digitalWrite(CS_PIN_MCP2515, HIGH);
digitalWrite(10, HIGH);

SPI.beginTransaction(SPISettings(5000000, MSBFIRST, SPI_MODE3));
SPI.begin();

initAdxl();
initMcp2515();
}
\end{lstlisting}
	%
	% Loop-Routine
	%
	\item \textbf{Loop-Routine}\\
	In der Loop-Routine (s. Listing \ref{lst:mcpExecutorLoop}) werden zwei Methoden aufgerufen, von denen die Prozedur \textit{getAdxlData()} die Sensordaten lie�t und zusammen mit einem Zeitstempel in ein Array speichert.
	Die Routine \textit{mcp2515\_load\_tx\_buffer()} schreibt die Daten nacheinander in die Sende-Buffer 1-3.\\
	Innerhalb der Routine \textit{getAdxlData()} erfolgt in Zeile 10 das Setzen der entsprechenden Registerwerte, um die Sensordaten auslesen zu k�nnen. In den Zeilen 12-15 erfolgt das Auslesen, indem der Selektor-Pin auf Low-Level gelegt wird. Anschlie�end erfolgt das Senden von 0-Werten, um per SPI die Sensordaten auszulesen (f�r jede Achse zwei Byte). In Zeile 17 wird noch die ID f�r das Ger�t angegeben.\\
	Innerhalb der Routine mcp2515\_load\_tx\_buffer0() erfolgt in den Zeilen 25-47 das Setzen der Registeradresse eines Sendebuffers, in Abh�ngigkeit welcher zuletzt genutzt wurde. In Zeile 51 werden in den ausgew�hlten Sendebuffer die Sensordaten gelegt und mit dem in Zeile 53 angegebenen Kommando auf den CAN-Bus geschickt.
	\label{lst:mcpExecutorLoop}
\begin{lstlisting}[language=Java, caption=Loop Routine]
void loop()
{
	getAdxlData();
	
	for (size_t i = 0; i < MESSAGE_SIZE_ADXL; i++) mcp2515_load_tx_buffer(ReadBuf[i], i, MESSAGE_SIZE_ADXL);
}
...
bool getAdxlData() {

	byte RegAddrBuf[] = { REGISTER_ACCEL_REG_X | REGISTER_ACCEL_SPI_RW_BIT | REGISTER_ACCEL_SPI_MB_BIT };
	
	setCsPin(CS_PIN_ADXL, LOW);
	SPI.transfer(RegAddrBuf[0]);
	for (int i = 0; i < 6; i++) ReadBuf[i] = SPI.transfer(0x00); 
	setCsPin(CS_PIN_ADXL, HIGH);
	
	ReadBuf[6] = EXECUTOR_ID;
	
	return true;
}
...
void mcp2515_load_tx_buffer(byte messageData, int byteNumber, int messageSize) {
	byte registerAddress;
	
	if (currentTxBuffer[0])
	{
	registerAddress = REGISTER_TXB0Dx[byteNumber];
	if (byteNumber == (messageSize-1))
	{
	currentTxBuffer[0] = false;
	currentTxBuffer[1] = true;
	}
	}
	else if (currentTxBuffer[1]) {
	registerAddress = REGISTER_TXB1Dx[byteNumber];
	if (byteNumber == (messageSize - 1))
	{
	currentTxBuffer[1] = false;
	currentTxBuffer[2] = true;
	}
	}
	else if (currentTxBuffer[2]) {
	registerAddress = REGISTER_TXB2Dx[byteNumber];
	if (byteNumber == (messageSize - 1))
	{
	currentTxBuffer[2] = false;
	currentTxBuffer[0] = true;
	}
	}
	
	mcp2515_execute_write_command(registerAddress, messageData, CS_PIN_MCP2515);
	
	for (int i = 0; i < 3; i++) if(currentTxBuffer[i] == true) mcp2515_execute_rts_command(i);	
}
\end{lstlisting}

\end{enumerate}