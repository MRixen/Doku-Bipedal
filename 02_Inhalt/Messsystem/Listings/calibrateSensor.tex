\label{lst:calibrateSensor}
\begin{lstlisting}[language=Java, caption=Sensoren kalibrieren]
private void button_calibrateSensors(object sender, EventArgs e)
{
	for (int i = 0; i < SENSOR_AMOUNT; i++)
	{
		gs_x[i] = 0; gs_y[i] = 0; gs_z[i] = 0;
		sensorCalibrationSet[i] = false;
	}
	buttonCalibrateSensors.Enabled = false;
}
...
private void tcpMsgRecEvent(string[] receivedMessage)
{
	...
	if (sensor_joint_ID <= SENSOR_AMOUNT)
	{
		if ((!sensorCalibrationSet[sensor_joint_ID]) & (!buttonCalibrateSensors.Enabled))
		{
			double alpha = -9999;
			if ((sensorValues[1] >= 0) & (sensorValues[2] < 0)) alpha = -(Math.PI / 2) + Math.Acos((sensorValues[1] * GRAVITATION_EARTH) / GRAVITATION_EARTH);
			else if ((sensorValues[1] >= 0) & (sensorValues[2] > 0)) alpha = (Math.PI / 2) - Math.Acos((sensorValues[1] * GRAVITATION_EARTH) / GRAVITATION_EARTH);
			
			if (alpha != -9999)
			{
				Rx_x[0] = 1;
				Rx_x[1] = 0;
				Rx_x[2] = 0;
				
				Rx_y[0] = 0;
				Rx_y[1] = Math.Cos(alpha);
				Rx_y[2] = -Math.Sin(alpha);
				
				Rx_z[0] = 0;
				Rx_z[1] = Math.Sin(alpha);
				Rx_z[2] = Math.Cos(alpha);

				sensorCalibrationSet[sensor_joint_ID] = true;
				
				for (int i = 0; i < SENSOR_AMOUNT; i++)
				{
					if (!sensorCalibrationSet[i]) break;
					else if (i == SENSOR_AMOUNT - 1) helperFunctions.changeElementEnable(buttonCalibrateSensors, true);
				}			
			}
			else
			{
				helperFunctions.changeElementText(textBox_Info, "Calibration failed!", true);
				helperFunctions.changeElementEnable(buttonCalibrateSensors, true);
			}
		}
	}
	...
}
\end{lstlisting}