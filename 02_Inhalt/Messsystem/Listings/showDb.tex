\label{lst:showDb}
\begin{lstlisting}[language=Java, caption=Datenbank anzeigen]
private void checkBox_showDatabase_CheckedChanged(object sender, EventArgs e)
{
	...
	DataSet dataSetTemp = new DataSet();
	int databaseId = Int32.Parse(textBox_dataBaseId.Text);
	
	if (((CheckBox)sender).Checked)
	{
		switch (databaseId)
		{
			case 1:
			dataSetTemp = dataSet_Db1;
			break;
			case 2:
			dataSetTemp = dataSet_Db2;
			break;
			default:
			dataSetTemp = dataSet_Db1;
			break;
		}
		dataBaseList = new DatabaseList(this, dataSetTemp, databaseConnection, databaseId);
		dataBaseList.Show();
	}
	else try dataBaseList.Close();
	...
}
\end{lstlisting}