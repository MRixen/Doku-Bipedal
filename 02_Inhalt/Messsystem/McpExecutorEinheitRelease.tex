\subsection{Release version}
\label{kap:McpExecutorReleaseversion}
Bei der Realease-Version der Mcp-Executor-Einheit (s. Abbildung \ref{fig:McpExecutorRelease}) werden die Beta-Versionen zun�chst bewertet und alle negativen Aspekte aufgezeigt und m�gliche L�sungen erarbeitet. Resultierend ist ein einfach und schnell zu bauendes Ger�t, welches zudem kosteng�nstig in der Anschaffung ist (s. Tabelle \ref{tab:kostenEinheit}).

\begin{table}[H]
\caption{Kosten einer Einheit}\label{tab:kostenEinheit}
\renewcommand{\arraystretch}{1.5} 
\newcolumntype{C}[1]{>{\centering\arraybackslash}p{#1}}
\centering
\begin{tabular}{|p{5cm}|p{5cm}|}
	\hline 
	\textbf{Artikel} & \textbf{Kosten \euro} \\ 
	\hline 
	Mcp2515 & 2.30 \\ 
	\hline 
	Adxl345 & 1.60 \\ 
	\hline 
	Geh�use & 0.90 \\ 
	\hline 
	Abdeckblech & 0.40 \\ 
	\hline 
	Platine & 0.30 \\ 
	\hline 
	Stift- Buchsenleiste & 0.99 \\ 
	\hline 
	Atmega328 & 2.20 \\ 
	\hline 
	Quarz mit Oscillator & 0.30 \\ 
	\hline 
	Klettverschluss &  \\ 
	\hline 
	& \textbf{8.99} \\ 
	\hline 
\end{tabular} 
\end{table}
	
Die Release-Version ist in Abbildung \ref{fig:McpExecutorRelease} dargestellt und dessen Komponenten werden in Tabelle \ref{tab:KomponentenReleaseVersion} erl�utert.
 
\begin{figure}[H]
	\centering
	\includegraphics[width=0.7\linewidth]{03_Grafiken/pseudoImage}
	\caption[Release-Version der Mcp-Executor-Einheit]{Release-Version der Mcp-Executor-Einheit}
	\label{fig:McpExecutorRelease}
\end{figure}

\begin{table}[H]
	\caption{Komponenten der Mcp-Executor Einheit (Release)}\label{tab:KomponentenReleaseVersion}
	\renewcommand{\arraystretch}{1.5} 
	\newcolumntype{C}[1]{>{\centering\arraybackslash}p{#1}}
	\centering
	\begin{tabular}{|p{1cm}|p{4cm}|p{5cm}|}
		\hline 
		\textbf{Pos.} & \textbf{Name} & \textbf{Beschreibung}\\ 
		\hline 
		1 & & \\ 
		\hline 
		2 & & \\ 
		\hline 
		3 & & \\ 
		\hline 
		4 & & \\ 
		\hline 
		5 & & \\ 
		\hline 
		6 & & \\ 
		\hline 
		7 & & \\ 
		\hline 
		8 & & \\ 
		\hline 
		9 & & \\ 
		\hline 
		10 & & \\ 
		\hline 
	\end{tabular} 
\end{table}

W�hrend der Serienfertigung wurden �nderungen an den Steckerkontakten, sowie den Leitungen durchgef�hrt, um ein Herausrutschen zu verhindern.