\chapter{Anwendung}
\label{kap:Anwendung}
Das System liegt in der \textbf{Version 1.0.0} vor. Das bedeutet, es wurde getestet und kann, wie unten beschrieben, verwendet werden. Fehler sind nicht ausgeschlossen und m�ssen dokumentiert und anschlie�end behoben werden. Des Weiteren erfolgt das kontinuierliche Verbessern der Soft- und ggf. Hardware hin zu einer neuen Version.

Um einen neuen Datensatz f�r eine Bewegung aufzuzeichnen, muss wie folgt vorgegangen werden:
\begin{enumerate}
	\item Mess-Anzug ankleiden und das technische Equipment, gem�� Abbildung XXX, anlegen und miteinander verbinden (Raspberry Pi zuletzt anschlie�en!).
	\item Software ClientGraph\_MovementDiagnose starten, Samples angeben und Button Record to db bet�tigen
	\item Messwerte exportieren und in Excel als CSV-Datei (Semikolon-Delimited) importieren
	\item Schrittfolge extrahieren (max. 65535 ms) und als CSV-Datei (Tab-Delimited) exportieren
	\item Trennzeichen Tab (mit Hilfe von Notepad++) durch Zeichen Semikolon ersetzen und ggf. �berschrift entfernen
	\item Alle Text-Dateien nacheinander mit der Software ClientGraph\_MovementDiagnose in die Cloud laden (Button Upload)
	\item Robotersystem ausrichten (gem�� Abbildung XXX) und das Programm MovementControl starten
\end{enumerate}
