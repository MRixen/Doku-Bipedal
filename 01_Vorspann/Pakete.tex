% Standardpakete
\usepackage[latin1]{inputenc}
\usepackage[T1]{fontenc}
\usepackage[english,ngerman]{babel} 
% For paragraph (change counter deepth)
\usepackage{titlesec}

%Zitate erm�gliche

\usepackage[babel,german=quotes]{csquotes}

% Abs�tze durch zus�tzlichen Leerraum trennen
\usepackage{parskip}
% Ansonsten wird kein zus�tzlicher Leerraum
% eingef�gt, daf�r aber die erste Zeile einger�ckt

% Farbe
\usepackage{xcolor}

% Blindtext (zum Testen von Formatierungen)
\usepackage{blindtext}

% Farbige Tabellen
\usepackage{colortbl}
%\usepackage{cite}
% Bibliography
%\usepackage{natbib}
\usepackage{bibgerm}
\bibliographystyle{abbrvdin}
% Show footnotes at bottom of page
\usepackage[bottom]{footmisc}

\usepackage{booktabs}
\usepackage{longtable}

\usepackage{multirow}

\usepackage{array,ragged2e}

% Abk�rzungsverzeichnis
\usepackage[]{acronym}

%% Tabellen
\usepackage{array}

% MATHEMATIK --------------
% Verbesserter Mathesatz
%Basispaket
\usepackage[intlimits,sumlimits,namelimits]{amsmath}
% Verbesserung und Bugfix gegen�ber amsmath
\usepackage{mathtools}
%Fette kursive mathematische Symbole
\usepackage[]{bm}
% -----------------------------

% GLEITUMGEBUNG --------------
%% Grafiken
% Grafiken einbinden
\usepackage{graphicx}
%\usepackage{pstricks,pst-2dplot,pst-pdf} 
% verbesserte Kontrolle �ber Gleitobjekte
\usepackage{float}
% Beschriftung der Gleitobjekte anpassen
\usepackage[
	font={small,sf},
	labelfont=bf,
	format=hang,
	width=0.9\textwidth
]{caption}
% -----------------------------

%Kopf_und_Fusszeile
%\usepackage{fancyhdr}
\usepackage[automark]{scrpage2}

%\usepackage{enumitem}