\begin{table}[H]
\caption{Ablauf: NAME}\label{app:UseCaseNAME2}
\renewcommand{\arraystretch}{1.5} 
\newcolumntype{C}[1]{>{\centering\arraybackslash}p{#1}}
\centering
\begin{tabular}{|p{1.5cm}|p{1.5cm}|p{9cm}|}
 	\hline 
    \multicolumn{2}{|l|}{\cellcolor{hellgrau}\textbf{Name}} & Manuelles Verfahren des Roboters \\     
    \hline
    \multicolumn{2}{|l|}{\cellcolor{hellgrau}\textbf{Ziel}} & Verfahren des Roboters mit Tasten \\
    \hline
    \multicolumn{2}{|l|}{\cellcolor{hellgrau}\textbf{Akteure}} & Bediener, Roboter \\   
    \hline
    \multicolumn{2}{|l|}{\cellcolor{hellgrau}\textbf{Trigger}} & Starten des Steuerungsmen�s durch Bet�tigen der Verbindungstaste  \\   
    \hline
    \multicolumn{2}{|l|}{\cellcolor{hellgrau}\textbf{Vorbedingung}} & Aktive Datenverbindung zum Roboter \\   
    \specialrule{2pt}{0pt}{0pt} 
    \rowcolor{hellgrau} \textbf{Schritt} & \textbf{Akteur} & \textbf{Ablauf} \\
    \hline
    1. & Person & Aktivieren des manuellen Betriebsmodus \\
    \hline
    2. & Person & Einstellen der gew�nschten Geschwindigkeit \\
    \hline
    3. & Person & Verfahren des Roboters in die gew�nschte Ausgangsorientierung \\
    \hline
    4. & Roboter & Abarbeiten der empfangenen Beschleunigungswerte \\
    \hline
\end{tabular} 
\end{table}