\begin{table}[H]
\caption{Ablauf: NAME}\label{app:UseCaseNAME1}
\renewcommand{\arraystretch}{1.5} 
\newcolumntype{C}[1]{>{\centering\arraybackslash}p{#1}}
\centering
\begin{tabular}{|p{1.5cm}|p{1.5cm}|p{9cm}|}
 	\hline 
    \multicolumn{2}{|l|}{\cellcolor{hellgrau}\textbf{Name}} & Verbindung einrichten \\     
    \hline
    \multicolumn{2}{|l|}{\cellcolor{hellgrau}\textbf{Ziel}} & Verbindung zum Roboter aufbauen \\
    \hline
    \multicolumn{2}{|l|}{\cellcolor{hellgrau}\textbf{Akteure}} & Bediener \\   
    \hline
    \multicolumn{2}{|l|}{\cellcolor{hellgrau}\textbf{Trigger}} & Starten der Anwendung FloriBot HMI \\   
    \hline
    \multicolumn{2}{|l|}{\cellcolor{hellgrau}\textbf{Vorbedingung}} & Flugmodus ist aktiv \\   
    \specialrule{2pt}{0pt}{0pt} 
    \rowcolor{hellgrau} \textbf{Schritt} & \textbf{Akteur} & \textbf{Ablauf} \\
    \hline
    1. & Person & Eingabe der Master-Adresse \\
    \hline
    2. & Person & Eingabe des Publisher-Topics \\
    \hline
    3. & Person & Eingabe des Subscriber-Topics \\
    \hline
    4. & Person & Eingabe des Node Namens im Optionsmen� \\
    \hline
    5. & Person & Setzen des Themes im Optionsmen� \\
    \hline
    6. & Person & Bet�tigen der Verbindungstaste zum Starten des Verbindungsaufbaus \\
	\hline 
	\rowcolor{hellgrau} & & \textbf{Erweiterung} \\
	7.a &  & Kein Verbindungsaufbau m�glich, wenn Flugmodus deaktiviert. Ein Hinweis wird ausgegeben. \\
    \hline
    7.b &  & Kein Verbindungsaufbau m�glich, wenn nicht alle Eingabefelder ausgef�llt sind. Ein Hinweis wird ausgegeben. \\
    \hline
    7.c &  & Kein Verbindungsaufbau m�glich, wenn Wlan-Schnittstelle deaktiviert ist. Ein Hinweis wird ausgegeben. \\
	\hline 
\end{tabular} 
\end{table}