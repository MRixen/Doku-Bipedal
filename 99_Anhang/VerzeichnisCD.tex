Dieses Verzeichnis zeigt die wichtigsten Bestandteile der Daten-CD. Dabei stellt das Zeichen + einen Ordner und das Zeichen - eine Datei dar.

\vspace{1cm}
\parbox{0cm}{\begin{tabbing}
x \= x \= x \= x \= x \= x \kill
- Datei1 \\
- Datei2 \\
+ Ordner1 \\
+ Ordner2 \\
\> + Unterordner1-Ebene1 \\
\>\> + Unterordner1-Ebene2 \\
\>\>\> + Unterordner1-Ebene3 \\
\>\>\>\> + Unterordner1-Ebene4 \\
\>\>\>\> + Unterordner2-Ebene4 \\
\>\>\>\> + ... \\
\>\> + Unterordner2-Ebene2 \\
\>\>\> - Datei \\
\>\> + Unterordner3-Ebene2 \\
\>\>\> + Unterordner2-Ebene3 \\
\>\>\> + Unterordner3-Ebene3 \\
\>\>\> + ... \\
+ Ordner3 \\
\> + Unterordner2-Ebene1 \\
\>\> + Unterordner4-Ebene2 \\
\>\>\> + Unterordner1-Ebene4 \\
\end{tabbing}}