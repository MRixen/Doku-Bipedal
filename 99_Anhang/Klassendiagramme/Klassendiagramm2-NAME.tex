%
% Bild der Klasse NAME
%
\begin{figure}[H]
	\centering
	\includegraphics[width=0.7\linewidth]{03_Grafiken/pseudoImage}
	\caption[Klasse NAME]{Klasse NAME}
	\label{fig:pseudoimage}
\end{figure}

%
% Details der Klasse NAME
%
\begin{center}
	\renewcommand{\arraystretch}{1.2} 
	\newcolumntype{C}[1]{>{\centering\arraybackslash}p{#1}}
	\centering
	\begin{longtable}{|p{2cm}p{12cm}|}
		\caption{Parameter der Klasse NAME}\label{tab:KlasseParameter-NAME}\\
		\hline 
		\multicolumn{2}{|l|}{\cellcolor{gray}\textbf{Attribute}} \\ 
		\hline 
		\multicolumn{2}{|l|}{Element-Name + Datentyp (z.B. Button)}\\
		& Beschreibung der Funktion \\ 
		\hline 
		\multicolumn{2}{|l|}{Element-Name + Datentyp (z.B. Button)} \\
		\multicolumn{2}{|l|}{Weitere Element-Namen + Datentyp (z.B. Button)} \\
		& Beschreibung der Funktion \\ 
		\hline 
		%
		% METHODEN
		%
		\multicolumn{2}{|l|}{\cellcolor{gray}\textbf{Methoden}} \\ 
		\hline 
		\multicolumn{2}{|l|}{Funktion-Name (z.B. protected void NAME())} \\ 
		& Beschreibung der Funktion  \\
		\hline 
		\multicolumn{2}{|l|}{Funktion-Name (z.B. protected void NAME())} \\
		& Beschreibung der Funktion \\ 
		\hline 
		%
		% SCHNITTSTELLEN
		%
		\multicolumn{2}{|l|}{\cellcolor{gray}\textbf{Schnittstellen}} \\ 
		\hline 
		\multicolumn{2}{|l|}{Schnittstellen-Name (z.B. public interface NAME())} \\ 
		\multicolumn{2}{|l|}{Funktion-Name (z.B. public void NAME())} \\ 
		& Beschreibung der Schnittstelle  \\
		\hline 
	\end{longtable} 
\end{center}